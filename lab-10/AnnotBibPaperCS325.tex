\documentclass[12pt,letterpaper]{article}

% just for the example
\usepackage{lipsum}
% Set margins to 1.5in
\usepackage[margin=1.5in]{geometry}

% for graphics
\usepackage{graphicx}

% for crimson text
\usepackage{crimson}
\usepackage[T1]{fontenc}

% setup parameter indentation
\setlength{\parindent}{0pt}
\setlength{\parskip}{6pt}

% for 1.15 spacing between text
\renewcommand{\baselinestretch}{1.15}

% For defining spacing between headers
\usepackage{titlesec}
% Level 1
\titleformat{\section}
  {\normalfont\fontsize{18}{0}\bfseries}{\thesection}{1em}{}
% Level 2
\titleformat{\subsection}
  {\normalfont\fontsize{14}{0}\bfseries}{\thesection}{1em}{}
% Level 3
\titleformat{\subsubsection}
  {\normalfont\fontsize{12}{0}\bfseries}{\thesection}{1em}{}
% Level 4
\titleformat{\paragraph}
  {\normalfont\fontsize{12}{0}\bfseries\itshape}{\theparagraph}{1em}{}
% Level 5
\titleformat{\subparagraph}
  {\normalfont\fontsize{12}{0}\itshape}{\theparagraph}{1em}{}
% Level 6
\makeatletter
\newcounter{subsubparagraph}[subparagraph]
\renewcommand\thesubsubparagraph{%
  \thesubparagraph.\@arabic\c@subsubparagraph}
\newcommand\subsubparagraph{%
  \@startsection{subsubparagraph}    % counter
    {6}                              % level
    {\parindent}                     % indent
    {12pt} % beforeskip
    {6pt}                           % afterskip
    {\normalfont\fontsize{12}{0}}}
\newcommand\l@subsubparagraph{\@dottedtocline{6}{10em}{5em}}
\newcommand{\subsubparagraphmark}[1]{}
\makeatother
\titlespacing*{\section}{0pt}{12pt}{6pt}
\titlespacing*{\subsection}{0pt}{12pt}{6pt}
\titlespacing*{\subsubsection}{0pt}{12pt}{6pt}
\titlespacing*{\paragraph}{0pt}{12pt}{6pt}
\titlespacing*{\subparagraph}{0pt}{12pt}{6pt}
\titlespacing*{\subsubparagraph}{0pt}{12pt}{6pt}

% Set caption to correct size and location
\usepackage[tableposition=top, figureposition=bottom, font=footnotesize, labelfont=bf]{caption}

% set page number location
\usepackage{fancyhdr}
\fancyhf{} % clear all header and footers
\renewcommand{\headrulewidth}{0pt} % remove the header rule
\rhead{\thepage}
\pagestyle{fancy}

% Overwrite Title
\makeatletter
\renewcommand{\maketitle}{\bgroup
   \begin{center}
   \textbf{{\fontsize{18pt}{20}\selectfont \@title}}\\
   \vspace{10pt}
   {\fontsize{12pt}{0}\selectfont \@author} 
   \end{center}
}
\makeatother

% Used for Tables and Figures
\usepackage{float}

% For using lists
\usepackage{enumitem}

% For full citations inline
\usepackage{bibentry}
\nobibliography*

% Custom Quote
\newenvironment{myquote}[1]%
  {\list{}{\leftmargin=#1\rightmargin=#1}\item[]}%
  {\endlist}
  
% Create Abstract 
\renewenvironment{abstract}
{\vspace*{-.5in}\fontsize{12pt}{12}\begin{myquote}{.5in}
\noindent \par{\bfseries \abstractname.}}
{\medskip\noindent
\end{myquote}
}



% Set Title, Author, and email
\title{Annotated Bibliography; CS325}
\author{O Chinbat \\ chinbato@cwu.edu}
\date{\today}

\begin{document}

\maketitle
\thispagestyle{fancy}

\section*{Cloud computing for startups and it’s benefit: }
Cloud computing has transformed how startups operate by providing a flexible, scalable, and cost-efficient solution to IT infrastructure needs. Historically, startups faced significant barriers, such as the high costs of physical servers and complex infrastructure management, which limited their ability to compete with larger companies. By offering a "pay-as-you-go" model, cloud computing eliminates the need for substantial upfront investments, allowing startups to save money and allocate resources more effectively. Its scalability enables businesses to adjust resource usage as they grow, ensuring efficiency and reducing financial risks. Additionally, cloud platforms offer built-in tools for collaboration, analytics, and data security, empowering startups to focus on innovation and accelerate their time-to-market. As a result, cloud computing has become an essential driver of growth and success for small businesses in today’s competitive landscape.

% There are four placeholder entries below for your Annotated Bibliography. Please remember that you will need at least five for your rough draft and ten for your final version.
\subsection*{\bibentry{source1}}
Cloud computing offers transformative benefits for startups, enabling them to compete with larger enterprises through cost efficiency, scalability, and innovation. And this source highlight how pay-as-you-go models reduce upfront IT costs, allowing startups to allocate resources more effectively, while also discussing risks such as vendor lock-in and security concerns.

\subsection*{\bibentry{source2}}
This source provides a detailed analysis of how enterprise companies can leverage cloud computing to cut costs and save time. By eliminating the need to maintain their own servers, these companies can significantly reduce expenses related to hardware, maintenance, and energy consumption while streamlining their IT operations. This reduction in overhead allows resources to be allocated more effectively, enabling a focus on core business activities. Additionally, companies can dynamically scale their IT infrastructure based on demand, optimizing resource usage and minimizing waste. Outsourcing server management to cloud providers also means that enterprise companies can reduce the size of their IT departments, leading to further savings in salaries and training costs. As a result, startup companies adopting cloud computing from the beginning can avoid the same mistakes made by larger corporations. They can implement best practices in cloud utilization, ensuring agility, cost-effectiveness, and smooth scalability, thereby avoiding common pitfalls that hinder growth and sustainability.

\subsection*{\bibentry{source3}}
Sharing resources is economically advantageous for companies. Major corporations like Google, Amazon, and Microsoft, for instance, offer their surplus resources to other businesses, creating a mutually beneficial scenario. This resource-sharing is naturally positive for all involved. Additionally, cloud computing has generated new economic opportunities and jobs, such as cloud solution architects and cloud engineers.

\subsection*{\bibentry{source4}}
The protection of user data is of utmost importance for businesses, making data privacy in cloud computing a top priority. This source underscores the significance of robust privacy measures and outlines various methods to ensure the safety of user information. Among these methods is two-factor authentication, which adds an extra layer of security by requiring two forms of verification before granting access. Additionally, major cloud service providers implement rigorous security protocols and advanced encryption techniques to safeguard data. These providers, such as Amazon Web Services (AWS), Microsoft Azure, and Google Cloud, invest heavily in security infrastructure to prevent unauthorized access, data breaches, and other potential threats. By leveraging these comprehensive security measures, businesses can protect their user data and maintain trust with their customers.

\subsection*{\bibentry{source5}}
Effective monitoring of resource usage is crucial for companies to understand and manage their consumption. This article provides a comprehensive guide on how to monitor resources efficiently using Amazon Web Services (AWS), which is one of the most popular cloud providers. It emphasizes the importance of keeping track of resource utilization to ensure optimal performance and cost-efficiency. The article details various monitoring tools and techniques available in AWS, such as CloudWatch, which helps in tracking metrics, setting alarms, and generating logs. By utilizing these tools, companies can gain insights into their resource usage patterns, identify potential issues early on, and make informed decisions to optimize their cloud infrastructure.

\bibliographystyle{apalike2} 
\nobibliography{bibtemp}


\end{document}
